\documentclass{article}
\usepackage{listings}
\lstset{language=Python, breaklines=true}

\begin{document}

\title{Code Documentation Generation with AI Agents}
\author{}
\date{}
\maketitle

\section*{Purpose}
This script automates the process of generating documentation for provided code using three different AI agents. The agents are designed to handle specific tasks such as content generation, formatting, and validation of LaTeX documents. The final documentation is saved in a specified directory.

\section{Classes and Functions}

\subsection{Agent Class}
The \texttt{Agent} class is initialized with a system prompt and a model name. It has two methods: 
\begin{itemize}
    \item \textbf{\_\_init\_\_}: Sets up the initial parameters for the agent, including the system prompt and model selection.
    \item \textbf{run}: Takes user input and sends it to the AI model to generate documentation based on the provided code.
\end{itemize}

\subsection{RandomClass Class}
The \texttt{RandomClass} class has:
\begin{itemize}
    \item \textbf{\_\_init\_\_}: Initializes with a placeholder for something unknown.
    \item \textbf{do\_thing}: A method that returns the value 1, without any clear purpose defined in the provided code.
\end{itemize}

\section*{Main Script Logic}
The script performs several steps:
\begin{enumerate}
    \item Reads the current file's path and contents.
    \item Initializes three AI agents with specific system prompts.
    \item Generates documentation for the provided code using the first agent.
    \item Formats the generated LaTeX content from the generator into a structured format.
    \item Corrects any errors or issues in the formatted LaTeX content.
    \item Saves the corrected LaTeX content to a specified output directory as a file named "documentation\_output.tex".
\end{enumerate}

\end{document}